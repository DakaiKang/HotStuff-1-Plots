% \usepackage[hidelinks]{hyperref}
\let\Bbbk\relax
\usepackage{amssymb}
% \IEEEoverridecommandlockouts
%% Theorem-like environments.
\usepackage{amsthm}
\usepackage{enumitem}
\usepackage{xspace}
\newtheorem{theorem}{Theorem}[section]
\newtheorem{proposition}[theorem]{Proposition}
\newtheorem{lemma}[theorem]{Lemma}
\theoremstyle{definition}
\newtheorem{example}[theorem]{Example}
\newtheorem*{assumption}{Assumption}

\usepackage{adjustbox}
%%
\newcommand{\Changed}[1]{{\color{blue}#1}}

\newcommand{\Paragraph}[1]{\emph{#1}}

%% SI units and transaction notation.
\usepackage[binary-units]{siunitx}
\DeclareSIUnit{\txn}{txn}
\DeclareSIUnit{\batch}{batch}
\sisetup{per-mode=symbol}

\graphicspath{{./DrawIO/}}

%% Fully-replicated systems.
\newcommand{\Replica}[1][r]{\MakeUppercase{#1}}
\newcommand{\Client}[1][c]{\MakeLowercase{#1}}
\newcommand{\Replicas}[1][r]{\mathfrak{\MakeUppercase{#1}}}
\newcommand{\ID}[1]{\mathop{\textsf{id}}(#1)}
\newcommand{\T}{\tau}
\newcommand{\ma}{\mathbf{a}}
\newcommand{\m}{\mathbf{m}}
\newcommand{\n}{\mathbf{n}}
\newcommand{\f}{\mathbf{f}}
\newcommand{\nf}{\mathbf{nf}}
\newcommand{\Instance}[1]{\mathcal{I}_{#1}}
\newcommand{\Primary}[1]{\mathcal{L}_{#1}}

%% Protocol description.
\newcommand{\Transaction}{T}
\newcommand{\SignMessage}[2]{\langle#1\rangle_{#2}}
\newcommand{\View}{v}
\newcommand{\rn}{\rho}
\newcommand{\Leader}[1]{\mathcal{L}_{#1}}
\newcommand{\StopOp}{\texttt{stop}}
\newcommand{\Stop}[2]{\StopOp(#1; #2)}
\newcommand{\Message}[2]{\textsc{#1},#2}
\newcommand{\Hash}[1]{\operatorname{digest}(#1)}

%% Protocol names
\newcommand{\Name}[1]{\textnormal{\textsc{#1}}}
\newcommand{\BFT}{BFT}
\newcommand{\PBFT}{\Name{PBFT}}
\newcommand{\PoE}{\Name{PoE}}
\newcommand{\HS}{\Name{HotStuff}}
\newcommand{\HSTwo}{\Name{HotStuff-2}}
\newcommand{\HSOne}{\Name{HotStuff-1}}
\newcommand{\sysname}{\Name{HotStuff-1}}
\newcommand{\PHsysname}{\Name{Streamlined TrailBlazer}}
\newcommand{\SHSOne}{\Name{Streamlined HotStuff-1}}
\newcommand{\BHSOne}{\Name{Basic HotStuff-1}}
\newcommand{\BHSTwo}{\Name{Basic HotStuff-2}}
\newcommand{\PHSOne}{\Name{Streamlined HotStuff-1}}
\newcommand{\PHSTwo}{\Name{Chained HotStuff-2}}
\newcommand{\CHS}{\Name{CHotStuff}}
\newcommand{\SBFT}{\Name{Sbft}}
\newcommand{\Narwhal}{\Changed{\Name{Narwhal-HS}}}
\newcommand{\RBFT}{\Name{Rbft}}
\newcommand{\RCC}{\Name{RCC}}
\newcommand{\SpotLess}{\Name{SpotLess}}
\newcommand{\CSpotLess}{\Name{CSpotLess}}
\newcommand{\MirBFT}{\Name{MirBFT}}
\newcommand{\ISS}{\Name{ISS}}
\newcommand{\RDB}{\Name{Apache ResilientDB}}
\newcommand{\Bitcoin}{\Name{Bitcoin}}
\newcommand{\Ethereum}{\Name{Ethereum}}
\newcommand{\MinBFT}{\Name{MinBFT}}
\newcommand{\MDLG}{DLG}
\newcommand{\Slotter}{Slotter}

\newcommand{\helper}{helper\xspace}
\newcommand{\Helper}{Helper\xspace}


\newcommand{\earlycap}{Early Finality Confirmation}
\newcommand{\early}{early finality confirmation}
\newcommand{\strongspec}{Prefix Speculation}


%% Shorthands.
\newcommand{\SH}[1]{\texttt{#1}}
\newcommand{\MAC}{\SH{MAC}}
\newcommand{\DS}{\SH{DS}}
\newcommand{\RDMS}{\SH{RDMS}}
\newcommand{\GST}{\SH{GST}}

%% Sub-protocols.
\newcommand{\Protocol}{\Name{P}}
\newcommand{\RVS}{\Name{RVS}}
\newcommand{\Chained}{\Name{Chained}}
\newcommand{\LSO}{\Name{LSO}}
\newcommand{\IQC}[1]{$inQC_{#1}$}
\newcommand{\EQC}[1]{$exQC_{#1}$}
\newcommand{\Cert}[2]{\llparenthesis#1\rrparenthesis_{#2}}
\newcommand{\Certificate}[1]{\mathcal{P}(#1)}
\newcommand{\CCertificate}[1]{\mathcal{C}(#1)}
\newcommand{\Pending}{\mathcal{T}}

%% Theory-throughput parameters
\newcommand{\Bandwidth}{\mathit{B}}
\newcommand{\TP}[1]{\mathit{T}_{#1}}
\newcommand{\SizeP}{\mathit{sp}}
\newcommand{\SizeQ}{\mathit{sq}}
\newcommand{\SizeN}{\mathit{sn}}
\newcommand{\SizeM}{\mathit{sm}}
\newcommand{\SizeT}{\mathit{st}}
\newcommand{\SizeS}{\mathit{ss}}
\newcommand{\Max}{\textnormal{max}}
\newcommand{\CMax}{\textnormal{cmax}}
\newcommand{\CPBFT}{\textnormal{c}\PBFT{}}

%% pseudocode parameters
\newcommand{\VoteQC}[1]{\mathit{VoteQC}_{#1}}
\newcommand{\SyncQC}[1]{\mathit{SyncQC}_{#1}}
%% Misc.
\newcommand{\abs}[1]{\lvert #1 \rvert}
\newcommand{\dsfrac}[2]{#1 / #2}
\newcommand{\union}{\cup}
\newcommand{\intersect}{\cap}
\newcommand{\difference}{\setminus}
\newcommand{\Concat}{\oplus}
\renewcommand{\div}{\operatorname{div}}
\newcommand{\lfref}[2]{Line~\ref{#1:#2} of Figure~\ref{#1}}
\newcommand{\subref}[2]{\ref{#1}.\ref{#1:#2}}
\newcommand{\alglineref}[2]{Figure~\ref{#1}, Line~\ref{#1:#2}}

\newcommand{\cmt}[1]{{\color{red} #1}}

\newcommand{\Good}{\cellcolor{green!20}}

%% SpotLess
\renewcommand{\v}[1][v]{#1}
\newcommand{\MName}[1]{\textsc{#1}}
\newcommand{\PP}{\mathbb{P}}
\newcommand{\PPproof}[1]{\operatorname{cert}(#1)}
\newcommand{\PPf}[1]{\operatorname{claim}(#1)}
\newcommand{\AcceptedSet}{\mathbb{CP}}
\newcommand{\SClaim}[1]{\operatorname{sclaim}(#1)}
\newcommand{\CProof}[1]{\operatorname{cproof}(#1)}
\newcommand{\Proposals}[1]{\operatorname{proposals}(#1)}
\newcommand{\Support}[2]{\operatorname{support}(#1, #2)}
\newcommand{\Precedes}[1]{\operatorname{precedes}(#1)}
\newcommand{\Depth}[1]{\operatorname{depth}(#1)}
\newcommand{\RVar}[2]{\textit{#1}_{#2}}
\newcommand{\digest}[1]{\operatorname{digest}(#1)}
\newcommand{\view}[1]{\operatorname{view}(#1)}
\newcommand{\timer}[1]{\textit{t}_{#1}}
\newcommand{\Share}[2]{\delta_{#1}^{\mathcal{#2}}}


\newcommand{\twat}{\textit{txns\_w\_assigned\_timestamp}}
\newcommand{\twoat}{txns\_wo\_assigned\_timestamp}
\newcommand{\hts}{highest\_timestamps}
\newcommand{\txnts}{transactions\_to\_send}
\newcommand{\lotm}{local\_timer}
\newcommand{\asts}{\emph{assigned timestamp}}
\newcommand{\lpat}{\emph{lowest possible assigned timestamp}}

%% Backup-continue counter
\newcounter{mycounterf}

\usepackage[noend]{algorithmic}
\usepackage{algorithm,float}
\makeatletter
\newenvironment{breakablealgorithm}
  {% \begin{breakablealgorithm}
   \begin{center}
     \refstepcounter{algorithmic}% New algorithm
     \hrule height.8pt depth0pt \kern2pt% \@fs@pre for \@fs@ruled
     \renewcommand{\caption}[2][\relax]{% Make a new \caption
       {\raggedright\textbf{\ALG@name~\thealgorithm} ##2\par}%
       \ifx\relax##1\relax % #1 is \relax
         \addcontentsline{loa}{algorithmic}{\protect\numberline{\thealgorithm}##2}%
       \else % #1 is not \relax
         \addcontentsline{loa}{algorithmic}{\protect\numberline{\thealgorithm}##1}%
       \fi
       \kern2pt\hrule\kern2pt
     }
  }{% \end{breakablealgorithm}
     \kern2pt\hrule\relax% \@fs@post for \@fs@ruled
   \end{center}
  }
\makeatother

%% Protocol/Algorithm environment
\usepackage{algorithmic}
\renewcommand{\algorithmiccomment}[1]{{\color{orange}$\triangleright$\textit{#1}}}
\newcommand{\GETS}{:=}
\newenvironment{myprotocol}{
    \hrule
    \footnotesize
    \smallskip
    \algsetup{linenosize=\footnotesize}
    \begin{algorithmic}[1]
        \newenvironment{algenumerate}{\begin{enumerate}}{\end{enumerate}}
        \newcommand{\SPACE}{\item[]}
        \newcommand{\TITLE}[2]{\item[] \textbf{\underline{##1}} (##2) \textbf{:}\\[2pt]}
        \makeatletter
            \newcommand{\EVENT}[1]{\STATE \textbf{event} ##1 \textbf{do}\begin{ALC@g}}
            \newcommand{\ENDEVENT}{\end{ALC@g}}
        \makeatother
        \newcommand{\HSPCEVENT}{\phantom{\textbf{event} }}
        \makeatletter
            \newcommand{\FUNCTION}[2]{\STATE \textbf{function} \Name{##1}(##2) \textbf{do}\begin{ALC@g}}
            \newcommand{\ENDFUNCTION}{\end{ALC@g}}
        \makeatother
}{
    \end{algorithmic}%
    \hrule
}
% \newenvironment{myprotocol}{
%     \hrule
%     \smallskip
%     \footnotesize
%     \algsetup{linenosize=\footnotesize}
%     \begin{algorithmic}[1]
%         \newcommand{\SPACE}{\item[]}
%         \newcommand{\BREAKLINE}{\\\hspace{1cm}}
%         \newcommand{\TITLE}[2]{\item[] \textbf{\underline{##1}} (##2) \textbf{:}\\[2pt]}
%         \makeatletter
%             \newcommand{\EVENT}[1]{\STATE \textbf{event} ##1 \textbf{do}}
%             \newcommand{\ENDEVENT}{ \STATE \textbf{end event}}
%         \makeatother
% }{
%     \end{algorithmic}
%     \smallskip
%     \hrule
% }
\newcommand{\spa}{\hspace{\algorithmicindent}}

%% Figures and plots.
\usepackage{tikz,pgfplots,pgfplotstable}

\usetikzlibrary{shapes,arrows,automata,positioning,cd}
\tikzset{
  spotlessedge/.style   = {black, ->, >=stealth},
}
\usepackage{xcolor}

\usetikzlibrary{arrows.meta,decorations.pathreplacing}
\tikzset{
    >=Stealth,
    smalltext/.append style={scale=0.7},
    dot/.style={circle,scale=0.35,draw=black,fill=black}
}

\usetikzlibrary{automata, positioning, arrows}
\usetikzlibrary{arrows.meta}
\tikzset{
    >=Stealth,
    plot/.append style={baseline,scale=0.475},
    label/.append style={font=\strut\footnotesize},
    dot/.style={circle,scale=0.35,draw=black,fill=black},
}
\pgfplotscreateplotcyclelist{mycyclelist}{
    black   ,every mark/.append style={solid,fill=\pgfplotsmarklistfill},mark=*\\ 
    red     ,every mark/.append style={solid,fill=\pgfplotsmarklistfill},mark=square*\\
    blue    ,every mark/.append style={solid,fill=\pgfplotsmarklistfill},mark=triangle*\\
    teal    ,every mark/.append style={solid,fill=\pgfplotsmarklistfill},mark=pentagon*\\
    brown   ,every mark/.append style={solid,fill=\pgfplotsmarklistfill},mark=halfsquare*\\ 
    orange  ,every mark/.append style={solid,fill=\pgfplotsmarklistfill},mark=halfcircle*\\
    violet  ,every mark/.append style={solid,fill=\pgfplotsmarklistfill,rotate=180},mark=halfdiamond*\\
}
\pgfplotscreateplotcyclelist{mycyclelistex}{
    black   ,every mark/.append style={solid,fill=\pgfplotsmarklistfill},mark=*\\ 
    red     ,every mark/.append style={solid,fill=\pgfplotsmarklistfill},mark=square*\\
}
\pgfplotsset{
    tick label style={font=\large},
    legend style={font=\Large,cells={anchor=west}},
    title style={font=\Large},
    label style={font=\Large},
    width=262.5pt,
    height=185pt,
    every axis/.append style={
        ylabel near ticks,
        xlabel near ticks,
        mark size=2.5pt,
        cycle list name=mycyclelist,
        font=\Large,
        y tick label style={
                        /pgf/number format/precision=1,
                        /pgf/number format/fixed,
                        /pgf/number format/fixed zerofill
                    }
    },
    barstyle/.append style={
        ybar,
        bar width={0.5cm},
        enlarge x limits=0.4,
        enlarge y limits={upper=0.025},
        ymin=0,
        xtick=data
    }
}

\newcommand{\plotTheoryTPut}[2]{\begin{tikzpicture}[plot]
        \begin{axis}[title={($\SI{#1}{\txn\per\text{batch}}$)},ylabel={Throughput (\si{\txn\per\second})},
            xlabel={Number of replicas ($\n$)},ymax=700000, ymin=0,mark size=1pt,legend columns=1, title style={font=\huge},
            xlabel style={font=\huge}, legend style={at={(0.6,0.85)},anchor=west}]
            \addplot table[x={n},y={spotless}] {#2};
            \addplot table[x={n},y={rvs}] {#2};
            \legend{$\TP{\SpotLess}$, $\TP{\RVS}$};
        \end{axis}
    \end{tikzpicture}}
    
    
\newcommand{\axisticksnode}{16, 22, 28, 34}

\newcommand{\ploteViewSync}[2]{\begin{tikzpicture}[plot, scale = 0.98]
    \begin{axis}[title={#1},ylabel={View Synchronization Time (\si{\second})},xlabel={Number of replicas ($\n$)},ymin=0,mark size=0.8pt,legend columns=2, xtick={\axisticksnode}, legend style={at={(0.03,0.9)},anchor=west}]
        \addplot table[x={n},y={hs}] {#2};
        \addplot table[x={n},y={rvs}] {#2};
        \legend{\Chained{} \HS{},\RVS{}};
    \end{axis}
\end{tikzpicture}}



